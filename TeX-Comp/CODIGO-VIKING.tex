\begin{multicols}{2}
	{\normalsize
Esta versão do Código Viking acrescenta diversas melhorias.

O preâmbulo está bem mais enxuto e organizado, foram eliminados todos os pacotes redundantes ou conflitantes. Também foram organizados conforme categorias e receberam os devidos comentários, explicando a função básica de cada um, afim de facilitar estudos mais aprofundados.

Também troquei todas as fontes por OpenSans (sugestão do Cadunico), melhorei as proporções entre tamanho de fonte, margens e distâncias em geral, em relação ao tamanho da página, que agora é proporcional ao tamanho das telas de notebook e celular.

Os tamanhos e proporções escolhidos também geram menos problemas, se caso for necessário ajustar a pagina às pressas, para uso em uma tela mais antiga, como é comum em palestras em lugares onde você nunca esteve (e por isso não conhece o equipamento). Ou ainda imprimir os documentos em papel A4, por exemplo, e precisar diminuir o tamanho das fontes, para economizar papel.

Este mesmo preâmbulo será utilizado em 3 documentos que considero importantes: 

1. \textbf{Be-A-Ba da Informática}: um guia para aulas de informática.

As melhorias desta versão permitem o uso de iteração, provas lógicas, conversão de números binários para decimais e importação de códigos em C, Shell e Python.

Desta forma, o Be-a-Bá da  Informática poderá fazer o que se propôs, quando publicado originalmente na Revista Espirito Livre: ensinar código binário e o básico de Ciência da Computação para crianças, através de recursos gráficos simples.

2.\textbf{ EFESTUS}: Sistema de Gerenciamento para Laboratórios de Informática.

Feito para atender especialmente escolas públicas, um sistema que funciona como um livro didático interativo, em formato hipertexto, onde a escola pode automatizar o cronograma de atividades em seus Laboratórios de Informática, conforme seu próprio planejamento, de acordo com os parâmetros do BNCC, bastando que tenha um Instrutor de Informática ou técnico familiarizado com LaTeX, HTML, GNU-Linux e Apache (fácil).

Conforme a escola adiciona ou altera o arquivo relacionado a um dia letivo qualquer (por exemplo, AULA-5-23.TEX, referente a 23a. aula do 5o. ano), o sistema Efestus gera automaticamente um HTML no servidor local Apache da escola.

Quando o aluno liga o computador, é direcionado para esta página, onde pode buscar a atividade numa lista, seja ela um texto, imagem, apresentação de slides, video, aplicativo na máquina local ou qualquer página web, a escolha do professor.

3. \textbf{Qualquer documento} para apresentação de projetos, palestras ou aulas em geral.

Divirtam-se, nobres guerreiros!}
\end{multicols}

\null
\pagebreak
