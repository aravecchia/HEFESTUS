\begin{multicols}{2}
{
\large
O \textbf{Sistema Efestus} foi escrito com base no \textbf{Código Viking}, projeto iniciado em 2015, durante o \textbf{Fórum \nobreak Goiano de Software Livre} (FGSL), por iniciativa minha, a partir do trabalho e das idéias do prof. \textbf{Dr. Ole Peter Smith}, da Universidade Federal de Goiania, com a generosa contribuição da Comunidade Latino Americana de Software Livre, e a finalidade de potencializar o \textbf{ABC da Informática}, outro projeto relacionado ao ensino de Computação, cujas bases haviam sido lançadas na \textbf{Revista Espirito Livre}, alguns anos antes.

Naquele encontro - \textbf{FGSL 2015} - manifestei meu descontentamento em relação à linguagem HTML, quando o professor Ole Smith me apresentou o \textbf{\LaTeX}.

Pude então ver de perto os códigos-fonte que a Universidade Federal de Goiania utiliza para gerenciar não só o processo educacional, como também o controle de frequência e as notas dos alunos.

Tudo de forma simples, rápida, fácil e automática.

Falamos exaustivamente também das dificuldades de se implantar o ensino de Computação em todas as escolas públicas do Brasil.

Então observei que, se um sistema em \LaTeX\space pode gerenciar os cursos de uma Universidade do tamanho da UFG, por quê não um simples Laboratório de Informática de uma escola municipal? 

Afinal, seria um sistema muito mais simples, com a vantagem de poder ser aplicado em larga escala: se resolvêssemos para o Laboratório de uma escola, resolveríamos para todos os Laboratórios de todas as escolas Brasil.

Por quê ninguem havia pensado nisso antes? - perguntei.

Resposta: o professor Ole havia pensado, há muito tempo.

Porém, dadas as dificuldades de se fazer Educação neste país, simplesmente ninguém havia ainda conseguido escrever este código.

Em 2018, tive a honra de reencontrar o professor Ole Smith na \textbf{Latinoware} - Fórum Latino Americano de Software Livre e Tecnologias Abertas, às vesperas da aprovação do BNCC de Tecnologia.

Na ocasião, pude apresentar ao \textbf{prof. Dr. Julio Cesar Neves}, o pai do Ensino de Computação no Brasil, o notável desenvolvimento dos alunos da Escola Professor Paulista, utilizando o \textbf{ABC da Informática}, gerenciado pelo \textbf{Sistema Efestus}.

Discutia-se, na época, se os Laboratórios de Informática deveriam ensinar Ciência da Computação, ou ser apenas mais um apêndice no processo de alfabetização.

Os alunos do Professor Paulista provaram, na ocasião, que crianças não apenas são capazes de dominar os computadores muito melhor e mais rápido que os adultos, mas também podem aprender programação em linguagem Shell e Python, além de fazer isso brincando!

Exercitando o \textbf{pensamento educacional} da maneira correta e nas fases corretas do desenvolvimento infantil, não só foram capazes de se alfabetizar alguns meses mais cedo, como também dominar a matemática muito melhor e mais rápido, do que se fossem estimuladas apenas pelos métodos tradicionais.

Melhor ainda: o aprendizado se tornou uma atividade divertida para as crianças, uma brincadeira, tanto quanto a Educação Física!

Diversas outras pessoas e comunidades contribuiram para que tais resultados fossem obtidos, e destaco aqui os professores Tiago Sodré, Jefer Dörr, Karina Menezes, Daniel Basconcelos, Débora Garofalo, Ana Diniz e Zoraide Sgarbi.

Também merecem destaque as contribuições do C3SL-UFPR, Hacker Club Cascavel, Raul Hacker Club, Portal Embarcados, Portal Labirito, Laboratório de Garagem, comunidade Debian-BR, comunidade \LaTeX-BR, Instituto Newton Braga e toda a Communidade Latinoware.}
\end{multicols}

\vfill
\pagebreak