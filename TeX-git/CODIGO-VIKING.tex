\large
\begin{multicols}{2}
	A ideia do Código Viking é ser a base \LaTeX \space de trabalho para qualquer professor, tanto para organizar seu plano de trabalho, como para preparar suas aulas, escrever seus livros e artigos, ou desenvolver suas palestras, de forma rápida, fácil, com uma diagramação perfeita, digna de um bom professor.

Assim, surgiu também a idéia do \textbf{Sistema Efestus}, que incorpora scripts Shell e Python ao sistema operacional, para disponibilizar o plano de aulas numa rede local e organizar o trabalho nos Laboratórios de Informática, em escolas públicas de todo o Brasil.

Ja o \textbf{ABC da Informática} é um plano de aulas pessoal, meu, que incorporei a este documento, como exemplo do funcionamento do Código Viking, e para testar o Sistema Hefestus em sala de aula.

Você pode utilizá-lo a vontade, para fins não comerciais, como ponto de partida para criar seu próprio plano de aulas ou gerenciar o Laboratório da sua escola.

Sinta-se à vontade para enviar dúvidas ou sugestões para aravecchia@gmail.com.

Copie, modifique (mantendo os direitos autorais), divulgue e compartilhe!

\vfill\null
\columnbreak

\subsection[Preâmbulo]{Preâmbulo}

\lstinputlisting[basicstyle=\ttfamily, style=LaTeXStyle, label=lst:LaTeXCode]{./HEFESTUS.tex}

\vfill
\columnbreak

\subsection[Capítulos do documento]{Capítulos do documento}

\lstinputlisting[style=LaTeXStyle, label=lst:LaTeXCode]{./CAPITULOS-HEFESTUS.tex}

\subsection[Pacote Listings para códigos-fonte]{Pacote Listings para códigos-fonte}

{
	\normalsize
	\begin{verbatim}
\lstinputlisting[style=LaTeXStyle,
 label=lst:ArduinoCode]
	{./TeX-git/ARDUINO-CODE.tex}

\lstinputlisting[style=LaTeXStyle,
label=lst:ShellCode]
{./TeX-git/SHELL-CODE.tex}

\lstinputlisting[style=LaTeXStyle,
label=lst:LaTeXCode]
{./TeX-git/LATEX-CODE.tex}

\lstinputlisting[style=LaTeXStyle,
label=lst:LaTeXCode]
{./TeX-git/PYTHON-CODE.tex}
\end{verbatim}
}

\subsection[Exemplos de códigos-fonte em pacote Listings]{Exemplos de códigos-fonte em pacote Listings}


\lstinputlisting[style=ArduinoStyle, label=lst:ArduinoCode]
{./TeX-git/Arduino-CODE.tex}

\lstinputlisting[style=C, label=lst:CCode]
{./TeX-git/C-CODE.tex}

\lstinputlisting[style=LaTeXStyle, label=lst:LaTeXCode]
{./TeX-git/LaTeX-CODE.tex}

\lstinputlisting[style=PythonStyle, label=lst:PythonCode]
{./TeX-git/Python-CODE.tex}

\end{multicols}
