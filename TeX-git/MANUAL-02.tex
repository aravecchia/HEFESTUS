%SHELL
% Template for documenting your Arduino projects
% Author:   Luis Jose Salazar-Serrano
%           totesalaz@gmail.com / luis-jose.salazar@icfo.es
%           http://opensourcelab.salazarserrano.com

%%%%% Template based in the template created by Karol Koziol (mail@karol-koziol.net)

%\linespread{1.3}
\lstdefinestyle{C}{
	language=sh,
	fillcolor=\color{white},
	backgroundcolor=\color{white},
	basicstyle=\small\ttfamily,
	breakatwhitespace=false,
	breaklines=true,
	captionpos=b,
	commentstyle=\color{gray},
	deletekeywords={...},
	escapeinside={\%*}{*)},
	extendedchars=true,
	frame=single,
	keepspaces=true,
	keywordstyle=\bfseries\color{orange},
	morekeywords={*,...},
	numbers=left,
	numbersep=2mm,
	numberstyle=\footnotesize\color{darkgray},
	rulecolor=\color{black},
	rulesepcolor=\color{black},
	showspaces=false,
	showstringspaces=false,
	showtabs=false,
	stepnumber=1,
	stringstyle=\color{orange},
	tabsize=3,
	title=\lstname,
	emphstyle=\bfseries\color{blue},
	framexleftmargin=0mm,
	framextopmargin=1mm
}

 
\begin{multicols}{2}
\normalsize
	
Supondo que você sabe instalar um sistema operacional GNU de Kernel Linux, sugerimos utilizar uma distribuição baseada em Ubuntu, Ubuntu Studio,  ou, de preferência um a distro especializada em educação, como Linux Educacional ou Zorin.

Distribuições que utilizam Desktops mais leves, como Xubuntu ou Lubuntu, são indicadas para computadores mais antigos, de forma a aproveitar melhor os recursos de memória e processamento.

Escolhida a distribuição, a primeira coisa a fazer é, óbviamente, instalar o sistema operacional, criando um usuário \textit{professor}, com privilégios de \textit{sudoer} ou \textit{admin}.

Feito isto, instalado o sistema \textit{default}, vamos trocar a senha de \textit{root}, para facilitar nosso trabalho.

No computador central do Laboratório (servidor, escolha aquele que tiver mais poder de processamento e memória, é nele que você vai passar os próximos anos da sua vida):

Acesse sua área de trabalho, com a senha de usuário administrador (criada durante a instalação), e abra o terminal, utilize o atalho Alt+F2. Na caixa que aparecer na tela, digite \textit{xterm}, e aperte Enter.

A próxima janela é o terminal, não precisa ter medo dele! Na verdade, com o tempo, você vai passar a gostar muito dele!

Digite no terminal:

\begin{lstlisting}
sudo passwd root
\end{lstlisting}

Digite a nova senha e aperte ENTER, 2 vezes, nesta ordem:

\begin{itemize}
	\item Digite a nova senha
	\item Aperte Enter
\item Digite a nova senha (de novo)
\item Aperte Enter (de novo)
\end{itemize}

Repare que, ao digitar a senha, nenhum caractere é escrito no terminal. Isto serve para proteger a nova senha.

Pronto. Agora você é ROOT, tem poderes supremos sobre a máquina.

Portanto, daqui em diante, lembre-se sempre: com grandes poderes, grandes responsabilidades você terá!

Por hora, vamos nos divertir com os novos super-poderes. Digite:

\begin{lstlisting}
su -
\end{lstlisting}

Aperte Enter, digite sua senha de \textit{root} e confirme, apertando ENTER.

Ótimo, agora você tem controle total do servidor!

Bem vindo, Padawan!

O próximo passo é instalar todos os aplicativos didáticos que os alunos utilizarão, bem como todas as ferramentas administrativas e protocolos de rede. Este script deve ajudá-lo:

\begin{lstlisting}
apt update
apt install openssh-server openssh-client apache tomcat9 arduino frozen-bubble debian-junior gcompriz mc synaptic libreoffice
\end{lstlisting}

Reinicie o servidor, afim de que todos os novos serviços e protocolos instalados sejam carregados corretamente.

Repita esta operação em todos os computadores, de todos os alunos.

Segue o link:
\end{multicols}
\begin{center}
	\Huge	\href{https://github.com/aravecchia/Documentos}{https://github.com/aravecchia/Documentos}
\end{center}
\null\vfill\pagebreak


