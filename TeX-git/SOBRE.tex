\large \begin{multicols}{2}
Todo Laboratório de Informática de Escola tem um problema: ninguém sabe direito o que fazer ali!

É uma realidade que está mexendo com a vida de todo mundo, não só dos pais e professores.

Isso está colocando em risco o futuro dos nossos jovens (e eu estou falando de futuro prá daqui a \textbf{2 anos}) e compromentendo o funcionamento da própria sociedade como um todo.

Quanta gente não passou aperto durante a pandemia, porque não achava um sobrinho que soubesse fazer um documento online, por exemplo?

E o que dizer dos empregos que esses jovens terão acesso, então?

O problema é que as escolas, muitas vezes, não fazem a menor idéia do que deve ser ensinado sobre Computação para as crianças, principamente nos anos iniciais da alfabetização, que são os mais importantes, porque é na primeira infância que os conceitos básicos de lógica podem ser ensinados corretamente.

\begin{itemize}
\item Joãozinho, pegue uma \{camiseta verde \textbf{e} vermelha\}.

\item Joãozinho, pegue uma \{camiseta verde \textbf{ou} vermelha\}.

	\item Joãozinho, pegue \{1 camiseta verde\} \textbf{e} \{1 camiseta vermelha\}.
	
		\item Joãozinho, pegue \{1 camiseta verde\} \textbf{ou} \{1 camiseta vermelha\}.
\end{itemize}

Notou a diferença, professor(a)?

Pensa comigo:

\begin{itemize}
	\item Você que ralou e estudou 4 anos de pedagogia, letras, história, biologia, sabe mesmo o fazer, pra sua turma de 4\textordmasculine\space ano:

		\begin{itemize}
			\normalsize
		\item entrar no google
	\item pesquisar sobre um determinado assunto
	\item baixar 4 ou 5 imagens pro HD
	\item colocar tudo isso num editor de texto
	\item escrever um texto
	\item diagramar o documento
	\item colocar tipos e tamanhos de fontes, margens, espaçamentos, etc.
	\item salvar o documento do aluno num lugar onde você possa encontrar depois
	\item imprimir 20 documentos
	\end{itemize}
\end{itemize}

Você consegue mesmo ensinar tudo isso, em \textbf{50 minutos}?

E você, \textbf{coordenador}, sabe dizer detalhadamente, qual o conteúdo de Computação que precisa ser ensinado, do 1\textordmasculine\space ao 9\textordmasculine\space ano, e quais são as ferramentas digitais que o Laboratório de Informática precisa disponibilizar pros alunos, aula por aula?

As escolas:

\begin{itemize}
	\item não conhecem as ferramentas que têm à disposição
	\item não sabem que essas ferramentas são todas gratuitas
	\item não têm um plano de aulas pro professor seguir
	\item não têm uma estratégia de ensino de computação
	\item não sabem lidar com a questão da cyber-segurança
	\item mal têm um planejamento (quando têm), que se adequa à realidade dos alunos e ao equipamento da escola
 \item e principalmente: não têm uma \textbf{ferramenta}, um software que organize toda essa informação, e gerencie as atividades no computador de cada aluno.
\end{itemize}

Além disso, nossos professores não tiveram o devido preparo prá lidar com a tecnologia, na escala que nós enfrentamos hoje.

Não tem como um professor, que ralou e estudou 4 anos de licenciatura em História, por exemplo, saber o mínimo de HTML (assim como eu também não sei nada sobre Vigotsky).

E não basta a escola ter um plano de aulas de Computação.

Hoje nós já estamos melhor, temos o \textbf{BNCC de Tecnologia}, que é um plano metodológico, mas \textbf{ainda não é um plano de aulas}.

É uma \textbf{orientação metodológica}: na prática, serve para que a escola possa criar seu plano de aulas, adaptado à sua própria realidade, conforme a legislação disponível \href{http://portal.mec.gov.br/index.php?option=com_docman&view=download&alias=241671-rceb001-22&category_slug=outubro-2022-pdf&Itemid=30192}{neste link}.

Além desse plano de aulas, para que ele funcione, a escola precisa de um software prá gerenciar tudo isso, no computador que está na mão de cada aluno!

Por isso, o professor \textbf{Ole Smith}, da UFG, me propôs este desafio: criar um banco de dados que a escola pudesse utilizar, para montar seu próprio plano de aulas pro Laboratório de Informática, que o próprio aluno pudesse acessar, do seu computador.

E que ainda o professor não precisasse se desesperar, prá fazer sua aula, num ambiente que ele não está acostumado, onde ele não foi devidamente treinado nem orientado prá trabalhar, que são os Laboratórios de Informática.

Assim surgiu o \textbf{ABC da Informática}, um plano de aulas de Computação, que eu venho desenvolvendo e testando com crianças do Ensino Fundamental, e adaptando conforme as diretrizes do BNCC a cada faixa etária, há mais de 10 anos.

Não é um plano de aulas perfeito, nem está completo ainda, mas é um plano de aulas que funcionou prá mim e pros meus alunos, e ficarei feliz se ele servir de ponto de partida para outros professores e instrutores de Informática.

Mas é um documento que qualquer Escola pode modificar e adaptar livremente à sua própria realidade e ao seu projeto pedagógico.

O \textbf{Sistema Efestus} serve prá gerenciar tudo isso, e fazer com que cada atividade chegue pra cada aluno, automaticamente, e na hora certa.

O professor têm à mão uma biblioteca inteira de textos, imagens links para diversas atividades, aplicativos, programas, tudo o que ele precisar para servir de apoio à sua aula.

O projeto é todo disponibilizado em \textbf{Software Livre}, com licença não comercial, e você pode baixá-lo neste link \href{https://github.com/aravecchia/HEFESTUS/blob/main/BNCC-Tecnologia.pdf}{\textbf{\Large aqui}}.

Lá você vai encontrar um PDF com toda a descrição do projeto, e um plano de aulas para Computação, do 1\textordmasculine\space ao 5\textordmasculine\space ano fundamental, que é a faixa etária que eu trabalho.

Mas você pode adaptá-lo a qualquer faixa etária, e também para qualquer disciplina, como Letras ou História, por exemplo, se sua escola tiver um técnico que conheça \href{https://www.latex-project.org}{\LaTeX}, HTML, Apache e Linux.

Também você vai encontrar diversos links para atividades, o código-fonte de todo o projeto, textos da Wikipedia, imagens, diagramas, códigos de programação prá você ensinar pros seus alunos, esquemas de experimentos com robótica educacional, e mais um monte de materiais que serão úteis, prá você ensinar computação pros seus alunos da maneira \textbf{certa}: a maneira do \textbf{Software Livre}, ensinando as bases do pensamento computacional.

Fique a vontade prá baixar o projeto, copiar, distribuir e modificar.

Mostre para o seu cordenador, para o diretor da sua escola e, se puder, compartilhe com outros professores, porque você já vai estar contribuindo muito com o ensino de Tecnologia na sua escola!

	\begin{center}
	
	\includegraphics[width=.4\linewidth]{./IMG-GIT/tux.png}

\end{center}
\vfill
\end{multicols}

\pagebreak