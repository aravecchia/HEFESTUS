\documentclass[landscape]{article}


\usepackage[utf8]{inputenc} % Permite o uso de caracteres acentuados
\usepackage[T1]{fontenc} % Define a codificação da fonte
\usepackage[brazilian]{babel} % Traduz alguns termos para o português brasileiro

\usepackage{tikz}
\usepackage{hyperref}
\usepackage{xcolor}
\usepackage{pagecolor}

% Define as margens
\usepackage[margin=1cm]{geometry}

\geometry{paperwidth=14in,paperheight=9in}

% Define a cor verde-oliva
\definecolor{verde-oliva}{RGB}{140, 138, 62}

% Define a cor de fundo da página
\pagecolor{verde-oliva}

\definecolor{azul-brasil}{RGB}{0, 85, 164}
\color{white}
\title{\Huge Exemplo de página web}
	\Huge
\begin{document}

	\maketitle
	
	\vspace*{.15\textheight}
	
	\begin{center}
		\begin{tikzpicture}[node distance=\dimexpr\marginparsep+\marginparwidth\relax, every node/.style={minimum width=3cm, minimum height=3cm, draw=black, fill=azul-brasil!100, inner sep=0.5cm, rounded corners=10pt, align=center, text=white, font=\Huge}]
		\node at (-6,0) (button1) {\hyperref[https://www.google.com/]{1}};
		
		\node[right of=button1] (button2) {\hyperref[https://www.facebook.com/]{2}};
		
		\node[right of=button2] (button3) {\hyperref[https://www.twitter.com/]{3}};
		
		\node[right of=button3] (button4) {\hyperref[https://www.instagram.com/]{4}};
		
		\node[right of=button4] (button5) {\hyperref[https://www.youtube.com/]{5}};
		\end{tikzpicture}

		
	\end{center}
	
\end{document}
