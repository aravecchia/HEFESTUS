\ThisCenterWallPaper{1}{./IMG/marvin.jpg}

\begin{multicols}{2}
\huge Não entre em pânico!

\vfill\null
\pagebreak
 
\normalsize

Eu sei dos tempos sombrios e caminhos tortuosos você trilhou, até chegar aqui, professor/professor(a)!

Sei também das tempestades que se apresentam no horizonte, e conheço o mal que aflige seu coração:

A chegada do novo \textbf{BNCC de Tecnologia}, que impõe o \textbf{ENSINO de COMPUTAÇÃO} em todas as escolas, a partir de 2023. (Leia \href{http://portal.mec.gov.br/index.php?option=com_docman&view=download&alias=241671-rceb001-22&category_slug=outubro-2022-pdf&Itemid=30192}{\textbf{aqui}}).

\vfill\null
\columnbreak

Mas não entre em pânico: \textbf{você não está sozinho(a)}!

Este material veio para te ajudar a enfrentar os desafios que a tecnologia colocou em seu caminho!

\textbf{Eu sei, eu sei, eu sei...}

Ja sei de tudo o que você vai dizer, professor(a).

Decor e salteado!

Sei que você vai reclamar da vida, do Universo e tudo mais, todo dia, na cabeça do instrutor de Informática... 

Mas o medo leva à raiva, e a raiva leva ao caminho sombrio da Força.

Você só vai encontrar dor e sofrimento, se continuar por aí, professor(a).

Computadores só funcionam de um jeito, pelo menos neste Universo:  por meio de códigos binários, \textbf{zero e um}, ponto.

\textbf{Simples assim}!

%Por isso, aquele "\textbf{jeitinho}", que as pessoas estão muitas vezes acostumadas a encontrar, em muitas situacões adversas, no dia-a-dia, este jeitinho simplesmente \textbf{não existe}, quando falamos de computadores.

O computador não é como um carro, que pode andar com o pneu quase cheio.

No computador, o "pneu"\space está \textbf{cheio} ou \textbf{não}!

Não existe meio termo, ou margem para negociação, nem jeitinho, muito menos norma burocrática maliciosa que possa mudar isso.

A menos que você possa alterar as leis da Física.

%Computadores só entendem código binario, e ponto: não têm desejos, emocões, nem qualquer tipo de empatia ou desprezo pela nossa mísera existência mortal.
%
%Seres fascinantes!
%
%Pois eles têm um lado sublime: são máquinas! Se não têm desejos ou sentimentos, portanto, são confiaveis e precisos, de uma forma que nenhum ser humano jamais sera!

%Não foi à tôa, que Sarah Connor confiou a segurança de seu filho John ao esterminador do futuro.

%E estão ficando cada vez mais espertos, muito espertos e incrivelmente rápidos!
%
%Talvez, justamente por isso, despertem tanto fascínio e horror em nós, humanos, demasiadamente humanos.

Computadores são máquinas, cujo funcionamento deve ser sempre muito preciso, e devidamente programado, \textbf{previamente}.

Eles respondem \textbf{somente} da forma que foram \textbf{programados} para responder.

Não espere um bolo de laranja ali, se você o alimentou com bananas!

E alguém ainda tem que programar a receita \textbf{antes} de ligar o "forno", \textbf{e por escrito}, ou também não tem "bolo" \space nenhum.

\textbf{A brincadeira acaba aí, com uma turma de crianças frustradas, e um(a) professor(a) desesperado(a), reclamando do Instrutor de Informática} que, acredite ou não, e o único que ainda está ali tentando te ajudar.

Mas nesta hora, professor(a), você só pode \textbf{reclamar com o Universo}, porque é assim que ele funciona.

Não mate o mensageiro: é assim que os elétrons se movem pelo tecido espaco-tempo, pelo menos neste Universo, \textbf{a menos que você possa alterar as leis da Física}.

Lhamento!

Funciona assim: \textbf{se} a professor(a) quer uma aula, esta aula tem necessariamente que ser programada por \textbf{alguém}, de um jeito ou de outro, e pasmem: \textbf{por escrito}!

Não é programar "qualquer joguinho": você tem que programar \textbf{UM} determinado jogo (ou UMA determinada atividade didatica ESPECÍFICA), todo dia, para cada fase de aprendizado, cada aula, numa sequencia precisa.

Fazendo as contas, são \textbf{384 aulas}, se considerarmos do 1\textordmasculine\space ano Fundamental ao 3\textordmasculine\space ano do ensino Médio.

\textbf{Boa sorte, coordenador(a)!}

\textbf{Divirta-se}!

Mas lembre-se de que, se é muito trabalho escrever o resumo de 384 aulas, mesmo que seja \textbf{copiando da caderneta}, digitar 2 milhões de linhas de código é trabalho \textbf{impossível}, por mais qualificado, nerd, gênio ou superdotado que seja o instrutor de Informática!

Mas calma, professor(a): \textbf{é para isso que os computadores servem}!
 
Desde que você os trate com carinho e o devido respeito, eles são capazes de fazer coisas incríveis, que farão os olhinhos dos seus alunos brilharem!
 
 Garanto: sua aula nunca mais será a mesma!

Pois bem, \textbf{é disTRo que trata este material}:

\textbf{Um Sistema de Gerenciamento para Laboratórios de Informática, que pode ser adaptado a qualquer disciplina, com qualquer Projeto Pedagógico, de qualquer Escola, nos termos do BNCC, e que sirva para criar, editar, gerenciar e distribuir todas as atividades didaticas, pela própria rede local da escola, de maneira facil, rápida, livre, gratuita, segura e \underline{automática}.
}

E tem que ser automático, seu Alexandre?

Tem, professor(a): não existe outro meio, mas é justamente aí que a mágica acontece!

É aí que uma turminha, mesmo aquela considerada das mais fracas, consegue se alfabetizar alguns meses mais cedo que a média, e dispara as notas no IDEB, contando a tabuada "nos dedinhos".

É o que fazem as grandes instituicões de ensino, na essência: criam uma lista, com todo o material didatico, e o disponibilizam de maneira organizada, numa página WEB.

\textbf{Simples assim!}

Parece complicado, mas qualquer instrutor de Informática de escola pública consegue implementar este sistema. 

A coordenação, o corpo de professores e a direção da Escola só precisam dizer quais recursos precisam, para cada aula, e escrever uma listinha resumida, copiando da caderneta mesmo, das 32 atividades anuais que são realizadas no Laboratório de Informática.

\textbf{E só precisa fazer isso uma única vez}!

Por exemplo: se são 32 semanas letivas ao longo do ano, a escola tem turmas do 1\textordmasculine ao 5\textordmasculine, e leva seus alunos para o Laboratório 1 vez por semana, \textbf{32 x 5 = 160 atividades}, certo?

Então, basta escrever uma lista, uma descrição curtinha do que o professor vai precisar em cada aula, junto com as imagens, links, textos ou o nome do aplicativo, que serão trabalhados em cada aula.

O instrutor de Informática só precisa daí "traduzir" \space essa lista para linguagem \LaTeX.

Ah! mas o instrutor disse que não sabe \LaTeX!

Tranquilo, se ele aprendeu HTML, Apache e um pouco de Shell Linux, vai tirar \LaTeX\space de letra.

O computador faz o resto: quando o instrutor chegar de manhã, basta ligar os computadores e pegar seu café, enquanto a professora busca os alunos.

Quando os alunos chegarem, a atividade de cada aula ja estará lá, prontinha para o professor(a) começar sua aula.

O \textbf{Sistema Efestus} oferece ao professor um catalogo gigantesco de atividades, bem \textbf{detalhadas e personalizadas}, que vai funcionar perfeitamente, mesmo numa escola pequena e com poucos recursos, e nos horários certos.

E não vai apenas criar um cronograma detalhado de aulas, mas gerenciar uma vasta coleção de imagens, arquivos, aplicativos, videos e links para internet.

Basta apertarmos o botão \textbf{power} de todos os PCs, logo pela manhã, e esperar que \textbf{/bin/bash} faca o restante.

Em 10 minutos, todos os computadores estarão prontos, e a aula programada para \textbf{hoje} estará a um clique de distância de todos os alunos, de todas as turmas.

E ainda sobrou tempo pro "tio"\space tomar um café.

Viram como é fácil?

\vfill\null
\columnbreak

\textbf{Obrigado, Comunidade Software Livre}!
%Pessoalmente, optei por utilizar \LaTeX e conjunto com HTML, C, Python e, óbviamente, Shell. 

A tecnologia entrou na escola sem pedir licença, da mesma forma que o fez em todas as outras esferas da sociedade.

E temos que lidar com ela: gostemos ou não, ela esta aí, em todos os lugares.

\textbf{Reclamem com Alan Turing}.

É um caminho sem volta: a datilografia não vai voltar para o curriculum, o mimeógrafo já está no museu, e a caderneta está prestes a ser aposentada.

Eu sei que isso te assusta, professor(a), afinal, você não foi preparado(a) para este mundo digital, que estamos vendo hoje.

Mas eu tenho um segredo pra te contar: \textbf{eu também não}!

Nasci em 1974, a escola que frequentei me preparou para um mundo industrial, mecânico, burocrático.

Tínhamos que decorar a tabuada, e usar a calculadora era uma heresia, punida com 50 chibatadas.

Afinal, quem poderia imaginar, nos anos 80, que hoje teríamos práticamente um computador em cada residência, e mais: no bolso de cada pessoa!?

Aliás, pouquíssimas pessoas da nossa geração de professores tiveram acesso à computação, antes de chegar na Faculdade, por um motivo simples: não havia computadores, naquela época.

E mesmo os professores mais jovens, que tinham um computador em casa, durante a infância, não tiveram o ensino formal de computação, não aprenderam sequer os rudimentos do \textbf{pensamento computacional}, durante sua alfabetização.

Isso nos coloca um grande problema: \textbf{como vamos ensinar aquilo que não aprendemos}?

%Pois encaremos os fatos: os poucos professores de hoje, no ensino fundamental e medio, que aprenderam o básico sobre computação, o fizeram por caminhos experimentais, incorretamente atrelados a uma determinada marca comercial de software, e geralmente na idade adulta.
%

Ou pior: aprendemos pela lógica errada, do jeito errado, com as ferramentas erradas, e na hora errada!

Certas habilidades só podem ser efetivamente desenvolvidas, como sabemos, na infância.

E o \textbf{pensamento computacional} é uma delas: você até pode aprender a programar, na fase adulta. Mas dificilmente chegará ao mesmo nível de alguém que aprendeu sobre algoritmos na infância.

Depois, quando seu vizinho vem reclamar "dessa juventude que não trabalha, não estuda e só fica o dia inteiro com o celular na mao", só posso dizer: \textbf{é claro}!

Só posso sair em defesa dos jovens \textbf{padawans}, afinal, não são eles que espalham virus e fake news, nem soltam \textit{nudes} nos grupinhos de família! 


%Estranho seria se não ficassem! Mas pelo menos eles não mandam nudes por engano no grupo de família (momento para risada do gênio do mal)!
%
%Nem avisam pro mundo inteiro que a casa esta desprotegida, não perdem as senhas, e sempre deixam o GPS compartilhado com alguém conhecido, quando estão fora de casa. 

Os mais jovens, os adolescentes de hoje, pelo menos, conseguem se virar relativamente bem, no mundo digital, mesmo não tendo tido quase nenhum ensino formal, em computação.

Poderia ser melhor, e verdade, se os "adultos"\space soubessem brincar...

Pois é! Acharam que eu não iria falar isso?

Os "adultos"\space são o grande problema, quando falamos do ensino de tecnologia nas escolas, especialmente nos anos fundamentais!

Quem está na faixa de 30 anos ou mais, geralmente não teve contato com a tecnologia tão cedo, mas hoje são professores e gestores escolares, dos mais variados níveis e... surpresa!

Responsáveis por implementar o ensino de tecnologia nas escolas, ensinar aquilo que não aprenderam!

Sobrou (novamente) para o professor!

E também o coordenador, o diretor, o gestor, o supervisor \textbf{mas}, no final das contas, e o \textbf{instrutor de Informática} quem "tem que dar um jeito".

Como se ele fosse a fada dos dentes!

Só que, como eu expliquei lá atras, quando falamos de computadores, só existe \textbf{um jeito}: programando da forma correta.

Não tem "jeitinho", lembram? O computador não pode andar com o pneu "meio cheio".

\textbf{Lhamento}.

Na prática, o instrutor de Informática é quem tem que resolver o problema, sempre de última hora, mal remunerado e sob as condicões mais adversas possíveis (prá não dizer absurdas).

Então, este material é também (e especialmente) para o instrutor de Informática que, acredito, tera grande prazer em abrir o nano em \textbf{/bin/bash}, editar uns arquivos em \LaTeX ou \textbf{Python}, e dizer:

"Não se desespere, professor(a)!

Não entre em pânico."

\textbf{Cuide bem deste arquivo, padawan.}

\textbf{Lembre-se: Saber é poder! Mas, com grandes poderes, jovem Padawan, grandes responsabilidades você terá.}
	
\textbf{E que a Força esteja com você!}

\vfill 
\columnbreak

Quando falamos de computação e tecnologia, dentro das escolas, não se trata apenas de ensinar o que é um bit ou um Byte, e sim uma \textbf{forma de pensar}, que não se pode aprender adequadamente, senão na infância.

Responda rápido: você tinha um Atari na mesa da sala, em casa, quando tinha 10 anos de idade?

Nem eu, mas vi o Atari chegar e conquistar toda uma geração de \textbf{nerds}.

Sim, o nerd, lembra dele?

Aquele garoto esquisito, sozinho no páteo da escola, que lidava com \textit{bullying} diariamente, sabem o nerd da escola?

Pois bem, o nerd já imaginava, lá nos anos 80, um mundo cheio de computadores. Queríamos carros voadores e naves espaciais, também.

Assim como você, o nerd também cresceu, mas aprendeu a gostar de computadores, e muitos nerds daquele tempo se tornaram professores de computação, ou instrutores de Informática, com a inglória tarefa de ajudar a Escola nessa travessia, de um método de ensino burocrático, para um método digital.

Só que o video-game deixou de ser coisa de nerd: hoje está nas mãos de quase toda criança, gostemos disso ou não.

Redes sociais, celulares, comércio eletrônico, documentos digitais, vigilância automatizada, reconhecimento de imagens, interligência artificial, carros autônomos, comércio global, tudo isso ja é rotina.

A indústria já vem ubstituindo seus funcionários por robôs e programas de computador, há um bom tempo.

Alguém aí conhece alguma datilógrafa? Pois bem, a profissão nem existe mais!

O site \href{www.code.org}{Code.org}, com cerca de 30 professores, tem 30 milhões de alunos, e avalia todos automaticamente, um por um, e no instante em que o aluno responde uma questão.

É uma nova Revolução Industrial, muito maior e assustadoramente mais rápida!

Portanto o tempo é curto, e o futuro dos jovens está sendo tracado agora, neste exato momento, numa velocidade antes inimaginável: TeraBytes por segundo.

\textbf{Bem vindos ao século 21}!

Estamos falando de computação quântica e inteligência artificial, portanto não há mais tempo a perder.

Este trabalho, o \textbf{Sistema Efestus}, se tornou necessário ha alguns anos, quando percebi que é impossível, do ponto de vista técnico, que um Laboratório de Computação,seja administrado segundo padrões e métodos do tempo das nossas bisavós, como ainda é hoje, na maioria das escolas.

Acontece que orientar nossos jovens, quanto à esta nova realidade que se impõe, é dever da Escola, e é tarefa pra ontem!

Porque os avanços tecnológicos estão cada vez mais rápidos, e as mudanças na sociedade também.

Desde costumes até o mercado de trabalho estão sofrendo mudanças cada vez mais rápidas, e quem não as acompanhar terá muitos problemas, num futuro próximo!

E, quando digo próximo, não estamos falando mais de décadas ou sequer de anos, como nas gerações passadas.

Meses podem fazer muita diferença no futuro dos nossos jovens, como vimos na epidemia de Covid, que não teria sido contida, não fossem pesados investimentos em educação tecnológica, de países como China, Alemanha e Suécia.

E, quando digo Escola, sabemos nas costas de quem a bomba tinha que cair, amiguinhos: nas nossas, claro, porque tudo cai nas costas do professor, sempre!

Mas como falei: não se desespere, não entre em pânico! Porque este material foi desenvolvido para ajuda-lo nessa travessia, do mundo burocrático para o digital.

É mais que um livro didático sobre Informática: é um sistema digital para o gerenciamento de todo tipo de aulas, seja de Computação, Artes, Letras ou Ciências.

Qualquer escola pode baixar, instalar em seus computadores e adaptar à sua realidade ou ao seu próprio projeto pedagógico.

Foi desenvolvido para ser uma simples página web na rede interna da escola, coisa que qualquer instrutor de Informática tem habilidades para montar, editar e administrar.

E, como foi desenvolvido em \LaTeX, fica fácil o professor ou cordenador enviar um arquivo texto para o instrutor, com o planejamento de cada aula, e em poucos minutos qualquer material pedagógico estará a distancia de um clique, disponível na tela de cada aluno, com dia e hora agendados.

\vfill

\pagebreak

\end{multicols}